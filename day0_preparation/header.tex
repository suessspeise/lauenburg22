

% SPRACHEN etc..:

\usepackage{textcomp} % \micro und \perthousand. hauptnutzen: gensymb nervt nichtmehr mit warnings...
\usepackage{verbatim} % für block comments mit: \begin{comment} .... \end{comment}

\usepackage[ngerman, english]{babel}
\selectlanguage{english}
\usepackage[utf8]{inputenc}

\usepackage{enumitem,amssymb} % for todo lists
% https://tex.stackexchange.com/questions/247681/how-to-create-checkbox-todo-list
\newlist{todolist}{itemize}{2}
\setlist[todolist]{label=$\square$}


\usepackage{emoji}
% list in here: https://mirror.informatik.hs-fulda.de/tex-archive/macros/luatex/latex/emoji/emoji-doc.pdf



\usepackage{hyperref}
\usepackage{xcolor}
\hypersetup{
    colorlinks=true,
    linkcolor={black!50!black},
    citecolor={blue!50!black},
    urlcolor={blue!80!black},
    pdftitle={Overleaf Example},
}
    
    

% BILDER
\usepackage{graphicx} % Erlaubt, dass der Text um Bild fließt, erlaubt Croppen, erlaubt resize von tabellen etc.
%\usepackage{subcaption}
%\usepackage{rotating} % für 90 gedrehte Bilder mit \begin{sidewaysfigure}[ht]....
\usepackage{pdfpages} % für Bilder im PDF format
\usepackage{import} % für bilder mit pfaden (oder so?). inkscape hat mir das gesagt
\usepackage{float}
\usepackage{subcaption} % subfig is evil!
\usepackage[section]{placeins} % provides \floatbarrier
\usepackage{wrapfig}


\usepackage{tabularx} % to set tablewidth: \begin{tabularx}{\textwidth}{X|l}



\usepackage{multicol} % multicol environment für mehrere spalten
\usepackage{geometry} % Für Einstellen der Seitenränder
    % normal:
%    \geometry{a4paper, top=10mm, left=15mm, right=15mm, bottom=10mm, headsep=00mm, footskip=10mm} % wide
%    \geometry{a4paper, top=20mm, left=15mm, right=15mm, bottom=10mm, headsep=00mm, footskip=12mm}

    \geometry{a4paper, top=20mm, left=20mm, right=20mm, bottom=20mm, headsep=00mm, footskip=12mm}
                        
\usepackage{todonotes}




\usepackage{listings} % code listings
\definecolor{cCodeString}{RGB}{42,0.0,255}
\definecolor{cCodeKeywordstyle}{RGB}{150,0,150}
\definecolor{cCodeComment}{RGB}{63,127,95}
\lstset{language=python,
    columns=fullflexible,
    basicstyle=\small\ttfamily,
    keywordstyle=\color{cCodeKeywordstyle}\bfseries,
    commentstyle=\color{cCodeComment},
    stringstyle=\color{cCodeString},
}
\definecolor{light-gray}{gray}{0.95}
\lstset{basicstyle=\linespread{1.1}\ttfamily\footnotesize,
    backgroundcolor=\color{light-gray}, xleftmargin=0.7cm,
    frame=tlbr, framesep=0.2cm, framerule=0pt,
} % background color for listings
% for a single word this works quite well:
\newcommand\code[1]{\textbf{\texttt{#1}}}
